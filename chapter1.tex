\chapter{Літературній огляд досліджуваних систем}
\section{Експериментальне та теоретичне дослідження електронної будови TiS2}
Після відкриття А. Геймом та К. Новосьоловим графену у 2004 році~\cite{Graphene}, знову зріс інтерес до двохвимірних матеріалів, зокрема до дихалькогенідів перехідних металів та їх інтеркаляційних сполук, які дуже сильно досліджувались у 80-х роках. Ці сполуки мають формулу MX$_{2}$ (X = S, Se, Te; M = Ti, Zr, Hf, V, Nb, Ta, Mo, W).

Дослідження електронної будови TiS$_{2}$ дали суперечливі висновки. З одного боку деякі розрахунки зонної структури приводять до непрямого перекриття p/d смуг між точками $\Gamma$ та $L$ в діапазоні від 0.2 до 1,5 еВ. Інші стверджують, що TiS$_{2}$ є вузько-щілинним напівпровідником. Експериментальні данні отримані за допомогою фотоемісії вказують на те, що поведінка даної сполуки схожа на напівпровідникову~\cite{semiconducter} або ж напівметалічну~\cite{semimetal}. Як відомо, TiS$_{2}$ має високу електронну провідність без зовнішнього легування, однак походження цієї високої провідності, будь то напівметал або сильно легований напівпровідник, обговорюється протягом декількох десятиліть, але деякі недавні GW, DFT розрахунки в поєднані з сканувальною зондовою мікроскопією, все ж таки, стверджують, що висока провідність обумовлена сильним самолікуванням~\cite{semimetal_or_semiconducter}. 

Щодо прикладного застосування, то даний матеріал успішно використовується у літій-іонних акумуляторах у якості катода. Коефіцієнт дифузії літію у TiS$_2$ порядку $10^{-8}-10^{-7}$ см$^2$/с, на один --- два порядки вище, ніж у широко використовуваних оксидних катодів. Недавні досліди показують, що комірки TiS$_2$ могуть зберігати більш ніж 50 \% початкової ємності після 35 років зберігання. Однією з важливих причин, чому TiS$_2$ був обраний як катодний матеріал для літій-іонних батарей, є те, що він має високу внутрішню електропровідність без зовнішнього легування. Це відрізняється від деяких інших популярних катодів таких, як LiFePO$_4$, для яких низька провідність була головною проблемою.

У 2003 році метод низькотемпературного газофазного синтезу(TiCl$_4$ + 2H$_2$S $\rightarrow$ TiS$_2$ + 4HCl), був успішно використаний для отримання нанотрубок на основі TiS$_2$~\cite{nanotube}. Аналіз їх морфології і структури показав, що трубки складаються з співвісних шарів сульфіду титану (відстань між шарами становить 0,57 нм) з атомним співвідношенням Ti:S =1:2, мають відкриті кінці, середні значення зовнішнього і внутрішнього діаметрів трубок складають ~20 - 30 і ~10 нм відповідно. У роботах~\cite{nanotube2,nanotube3} були вивчені процеси інтеркалювання нанотрубок TiS$_2$ літієм і воднем і обговорені можливості їх використання в якості матеріалів для водневих акумуляторів. Матеріалознавчі перспективи різних класів наноструктур багато в чому визначаються їх електронними властивостями, які можуть істотно відрізнятися від відповідних кристалічних (3D) фаз. Зазначені властивості в свою чергу залежать від атомної будови і геометрії наноструктур. Так, нанотрубки дисульфідів Mo, W є напівпровідниками, причому в залежності від діаметра і атомної конфігурації стінок (так званої хіральності) величина забороненої щілини різко змінюється. Навпаки, всі NbS$_2$-нанотрубки за своїми провідними властивостями є металами~\cite{nanotube4,nanotube5,nanotube6}.