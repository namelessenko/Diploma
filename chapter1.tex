\chapter{Вступ}
Після відкриття А. Геймом та К. Новосьоловим графену у 2004 році \cite{Graphene}, знову зріс інтерес до двохвимірних матеріалів, зокрема до дихалькогенідів перехідних металів (Transition metal dichalcogenides) та їхніх інтеркаляційних сполук, які дуже сильно досліджувались у 80-х роках. Ці сполуки мають формулу MX$_{2}$ (X = S, Se, Te; M = Ti, Zr, Hf, V, Nb, Ta, Mo, W). TMDs складаються з шарів халькогену, перехідного металу, халькогену (X-M-X) і мають гексагональну або тригональну симетрію. Ці шари утримуються разом слабкими силами ван-дер-Ваальса, що дозволяє їх відокремлювати і осаджувати ці шари на різні підкладки. Цей простий метод отримання наноструктур у поєднанні з різноманітними фізичними властивостями TMD роблять їх перспективними матеріалами для спінтроніки, наноелектроніки, виробництва відновлюваної енергії, біохімічних застосувань, а також для valleytronics – абсолютно нового підходу до квантових обчислень.

Ключовою особливістю, загальною для багатьох матеріалів TMD, є впорядкування хвилі щільності заряду (CDW), яка на фазовій діаграмі часто знаходиться поблизу надпровідної фази. Типовим прикладом є 1T-TiSe$_2$, в якому перехід у стан CDW відбувається при температурі приблизно $\approx$ 200K. Нижче T$_{CDW}$ TiSe$_2$ кристалізується в форму надрешітки 2x2x2 з симетрією P-3c1 (165 просторова симетрія), в той час, як неспотворений кристал при T > T$_{CDW}$ має симетрію P-3m1 (164 просторова симетрія). Нижче температури переходу питомий опір сильно зростає.

Споріднені сполуки 1T-TiS$_2$ і 1T-TiTe$_2$ отримують з TiSe$_2$ шляхом ізовалентної заміни Te і S на Se відповідно. Сполуки мають ту ж кристалічну структуру, що і 1T-TiSe$_2$ за нормальних умов. Однак, на відміну від TiSe$_2$, в об'ємному 1T-TiTe$_2$ не спостерігається хвиля зарядової щільності. Нещодавно було показано, що (2x2) CDW може з'являтися тільки в моношарах TiTe$_2$. Експериментальна ситуація з TiS$_2$ більш суперечлива. Досі обговорюється, чи має він напівпровідникову або напівметалічну природу \cite{Dolui_2016}. Частина фазової діаграми TiSe$_2$--$_x$S$_x$ до X < 0,34 була нещодавно вивчена за допомогою ARPES і STM автори бачили \cite{Mottas_2019}, що CDW поступово пригнічується. Екстраполюючи свої результати, автори дійшли висновку, що CDW звертається в нуль при легуванні, близькому до x = 1.

Оскільки модуляція щільності заряду та спотворення структури в стані CDW відбуваються одночасно, рушійна сила переходу CDW все ще обговорюється. Було запропоновано кілька механізмів, включаючи фононну конденсацію \cite{PhysRevLett.86.3799}, екситонний механізм \cite{PhysRevLett.106.106404}, зонну нестійкість Яна-Теллера \cite{doi:10.1143/JPSJ.49.898}, орбітальне впорядкування \cite{PhysRevLett.107.176404}, взаємодія нестійкостей Купера і частинок-дірок \cite{PhysRevLett.113.177001}.

Метою даної роботи є вивчення зонної структури основного стану сполук TiS$_2$ TiSe$_2$ TiTe$_2$ за допомогою meta-GGA функціоналів з включенням Хаббардовської взаємодії. 

\section{Літературній огляд досліджуваних систем}
\subsection{TiS$_2$}
Дослідження електронної будови TiS$_{2}$ дали суперечливі висновки. З одного боку, деякі розрахунки зонної структури приводять до непрямого перекриття p/d смуг між точками $\Gamma$ та $L$ в діапазоні від 0.2 до 1,5 еВ. Інші стверджують, що TiS$_{2}$ є вузько-щілинним напівпровідником. Експериментальні дані, отримані за допомогою фотоемісії вказують на те, що поведінка даної сполуки схожа на напівпровідникову \cite{semiconducter} або ж напівметалічну \cite{semimetal}. Як відомо, TiS$_{2}$ має високу електронну провідність без зовнішнього легування, однак походження цієї високої провідності, будь то напівметал або сильно легований напівпровідник, обговорюється протягом декількох десятиліть, але деякі недавні GW, DFT розрахунки в поєднані з сканувальною зондовою мікроскопією, все ж таки, стверджують, що висока провідність обумовлена сильним самолегуванням \cite{semimetal_or_semiconducter}. 

Щодо прикладного застосування, то даний матеріал успішно використовується у літій-іонних акумуляторах у якості катода. Коефіцієнт дифузії літію у TiS$_2$ порядку $10^{-8}-10^{-7}$ см$^2$/с, на один-два порядки вище, ніж у широко використовуваних оксидних катодів. Недавні досліди показують, що комірки TiS$_2$ можуть залишати більш ніж 50 \% початкової ємності після 35 років зберігання \cite{2015JPS...280...18P}. Однією з важливих причин, чому TiS$_2$ був обраний як катодний матеріал для літій-іонних батарей, є те, що він має високу внутрішню електропровідність без зовнішнього легування. Це відрізняється від деяких інших популярних катодів, таких як LiFePO$_4$, для яких низька провідність була головною проблемою.

У 2003 році метод низькотемпературного газофазного синтезу (TiCl$_4$ + 2H$_2$S $\rightarrow$ TiS$_2$ + 4HCl), був успішно використаний для отримання нанотрубок на основі TiS$_2$ \cite{nanotube}. Аналіз їхньої морфології і структури показав, що трубки складаються з співвісних шарів сульфіду титану (відстань між шарами становить 0,57 нм) з атомним співвідношенням Ti:S =1:2, мають відкриті кінці, середні значення зовнішнього і внутрішнього діаметрів трубок складають  20 - 30 і  10 нм відповідно. У роботах \cite{nanotube2,nanotube3} були вивчені процеси інтеркалювання нанотрубок TiS$_2$ літієм і воднем і обговорені можливості їхньої використання в якості матеріалів для водневих акумуляторів. Матеріалознавчі перспективи різних класів наноструктур багато в чому визначаються їхними електронними властивостями, які можуть істотно відрізнятися від відповідних кристалічних (3D) фаз. Зазначені властивості, в свою чергу, залежать від атомної будови і геометрії наноструктур. Так, нанотрубки дисульфідів Mo, W є напівпровідниками, причому в залежності від діаметра і атомної конфігурації стінок (так званої хіральності) величина забороненої зони різко змінюється. Навпаки, всі NbS$_2$-нанотрубки за своїми провідними властивостями є металами \cite{nanotube4,nanotube5,nanotube6}.
\subsection{TiSe$_2$}
Діхалькогеніди перехідних металів групи IVB, такі як TiSe$_2$, кристалізуються в шаруваті квазідвовимірні структури, в яких перехідний метал титану октаедрично координується шістьма атомами халькогену, так званою структурою 1T. Послідовні "сендвічі" Se-Ti-Se з ковалентно-іонними зв'язками розділені зазором ван-дер-Ваальса, що є причиною високої анізотропії і великої стабільності поверхні [001] на слебі. Незважаючи на те, що існує гарне загальне розуміння електронної структури цих матеріалів, все ще залишаються відкритими питання, наприклад, чи утворюють стехіометричні сполуки, отримані з Ti, напівметали або непрямі напівпровідники при кімнатній температурі. Наприклад, відомо, що TiTe$_2$ утворює півметал з невеликим перекриттям між смугами, отриманими з Te-5p і Ti-3d \cite{PhysRevB29, PhysRevB54}, складатиме близько 600 меВ. TiS$_2$, з іншого боку, являє собою непрямий напівпровідник з невеликим зазором близько 300 меВ між відповідними смугами \cite{PhysRevB16, PhysRevB21}. Найбільш складним з'єднанням цього сімейства є TiSe$_2$, розташоване між TiTe$_2$ і TiS$_2$. Оскільки селен менш електронегативний, ніж сірка, очікується, що ширина забороненої зони в TiSe$_2$ менше, ніж у TiS$_2$, або навіть відсутня. Розрахунки зонної структури і вимірювання фотоемісії з кутовим розіленням \cite{PhysRevB17,PhysRevL55,SolidStateCommun53,PhysRevB61} призвели до того, що TiSe$_2$ являє собою півметал з невеликим перекриттям між максимумом валентної зони в центрі зони Бріллюена $\Gamma$ і мінімумом зони провідності на границі зони Бріллюена $L$. Цей результат підтверджується новітніми експериментами з оптичної спектроскопії. Однак вимірювання фотоемісії у 2002 р. \cite{PhysRev65,PhyRevLet88} показали, що існує дуже маленька заборонена зона між $A$ і $L$. Проблема такого аналізу полягає в тому, як дослідити незайняту зону провідності і визначити її мінімум і ширину забороненої зони за допомогою фотоемісійної спектроскопії.

Ця проблема може бути вирішена шляхом заповнення найнижчої зони провідності TiSe$_2$ електронами, щоб зробити її вимірюваною для фотоемісії, наприклад,  шляхом теплового заселення або за допомогою фізичної абсорбції полярних молекул на вандерваальсівську поверхню. Дійсно, такий ефект спостерігався при адсорбції H$_2$O на поверхні TiSe$_2$. Випромінювання Ti-3d дуже чутливе до впливу води. Якщо TiSe$_2$ є напівпровідниковим, така зміна заповнення смуг повинна бути викликана вигином смуги, спричиненим поверхневим диполем, тобто накопиченням або виснаженням поверхневого шару носіями і інвертуванням або навіть виродженням поверхні напівпровідника.
\subsection{TiTe$_2$}
Багато матеріалів MX$_2$ демонструють переходи з хвильовою щільністю заряду (CDW), але це не відноситься до об'ємного TiTe$_2$ \cite{PhysRevB29,TiTe2_2, TiTe2_3, TiTe2_4}. Цікавим контрастним випадком є пов'язаний з ним матеріал TiSe2, який демонструє об'ємний (2х2х2) перехід CDW при 205K \cite{TiTe2_5}. Об'ємний TiSe2 являє собою непрямий напівпровідник з крихітним зазором, що розділяє зони провідності і валентності \cite{TiTe2_6}. Незважаючи на відсутність відповідного вкладення поверхні Фермі, крихітний непрямий зазор може опосередковувати взаємодію Яна-Теллера або екситонну взаємодію, яка може призвести до переходу CDW \cite{TiTe2_6}. Навпаки, зв'язок в TiTe$_2$ менш іонний, ніж в TiSe$_2$. Зазор повинен істотно відрізнятися, і насправді матеріал являє собою метал або напівметал з негативною шириною забороненої зони близько -0.8 $еВ$ \cite{PhysRevB.54.2453}. Однак отримані поверхні Фермі не мають областей, придатних для нестінгу. Таким чином, в TiTe$_2$ не очікується і не спостерігається CDW відповідно до традиційної картини. Як TiTe$_2$, так і TiSe$_2$ складаються з шарів, вільно укладених один на одного, і при переході від 3D до 2D не очікується ніяких значних електронних ефектів. Дійсно, одношаровий TiSe$_2$ показує перехід CDW всього на ~27K до вище температури об'ємного переходу \cite{TiTe2_7}.
Дивно, що дослідження засноване на фотоемісійної спектроскопії з кутовим розділенням (ARPES) і скануючої тунельної мікроскопії та спектроскопії (STM / STS), показує, що одношаровий TiTe2 демонструє перехід CDW (2 × 2), але двошаровий і багатошаровий TiTe2 не показують пов'язаних переходів. Одношаровий TiTe2, мабуть, ілюструє появу нової фізики в 2D-вимірі. Аномальна поведінка одного шару TiTe2 ставить під сумнів більш широку проблему механізмів CDW в цілому.
\section{Висновки до розділу} 
У цьому розділі зроблено вступ до тематики дипломної роботи та зроблено огляд літератури. Можна сказати не зважаючи на те, що самі сполуки дихалькогенів дуже сильно досліджувались у 60-70 роках, залишається багато відкритих питань або спірних моментів. 
Сполуки дихалькогеніду титану TiX2 (X = S, Se, Te) були широко вивчені через їх ні цікаві структурні та електронні властивості. TiX$_2$ демонструють великий потенціал для різних технологічних застосувань \cite{Yoffe}. Ці сполуки складаються в основному з гексагонального шару атомів Ti, затиснутого між двома аналогічними шарами атомів халькогену (X), утворюючи  сендвіч X-Ti-X. Шари пов'язані відносно слабкими силами ван-дер-Ваальса, в той час, як атоми всередині шару пов'язані сильними ковалентними зв'язками. Ці сполуки мають високі анізотропні фізичні властивості настільки, що їх можна розглядати, як двовимірні тверді тіла. В результаті цього TiX$_2$ може бути інтеркальований з сторонніми атомами і молекулами, що призводить до значних змін їх електронних властивостей і робить їх технологічно корисними. Наприклад, TiS$_2$ інтеркальований Li знайшов застосування в літієвих батареях.

Існує багато досліджень з приводу електронної будови TiX$_2$. Умрігар та інш. \cite{Benesh_1985} розраховували за допомогою LAPW (self-consistent linear augumented plane-wave) схеми, показали що TiS$_2$ це напівметал. Це не сильно узгоджувалось з оптичними експериментами Грінвея та Ніцше \cite{GREENAWAY19651445}, де вони показали що TiS$_2$ має напівпровідникову щілину 1-2 eV. Ліан та інші. \cite{Beal_1972} теж не змогли усунути це неспівпадіння. Кліпштейн та Фрєнд заключили \cite{Klipstein_1984} перекриття між S $3p$ станами та Ti $3d$ зростає зі значеннями 4.5 meV/kbar. Ця робота також стверджує що TiS$_2$ є напівпровідником з щілиною 0.18 eV. 

Вимірювання поляризації рентгенівської абсорбції біля границь спектру \cite{PhysRevB.58.7668,PhysRevB.56.3212, PhysRevB.8.3576} та інфрачервоний спектр сполук у CDW стані \cite{PhysRevB.29.2060} ще більш сприяли до зростання інтересу до цих сполук. Классен та інш. \cite{PhysRevB.54.2453} та де Боер та інш. \cite{PhysRevB.29.6797} вивчали електронну будову 1T-TiTe$_2$  за допомогою спектроскопії з кутовим розділенням (ARPES) та знайшли та узгодження з функціональними розрахунками електронної будови. Результати узгоджуються напівметалеву природу матеріалу з перекриттям у Te 5$p$ та Ti $3d$ смугами. Кід та інш. досліджували електрону будову TiSe$_2$ щоб виявити природу утворення (2$\times$2$\times$2) CDW переходу \cite{PhysRevLett.88.226402}.

В даній роботі буде представлено інший підхід до розрахунків електронної будови TiX$_2$ матеріалів, а саме буде випробувано метод, відомий як, строго обмежений і відповідним чином нормований функціонал з доданою взаємодією Хаббарда (SCAN+U). 

