\chapter*{Висновки}
%\addcontentsline{toc}{chapter}{Висновки} 
Методом DFT для сполук 1T-TiX$_2$ (X = S, Se, Te) показано, що вони мають напiвметалеву природу, окрiм TiS2. Для сполук TiTe$_2$ , TiSe$_2$ встановлено перекриття мiж смугами $p$ i $d$ халькогену та титану вiдповiдно. Спостерігається збiльшується DOS бiля E$_F$ при переходi вiд S до Te, Se, що обумовлено перекриттям Ti $d$ та Te, Se $p$ орбiталей у даних сполуках у значно бiльшій мірі нiж у сполуці з S.

Енергетична щiлина у сполуці TiS$_2$ задовільно описується тільки з урахуванням взаємодiї Хаббарда. Оптимальне значення U=2.1 eV вiдкриває щілину до 0.1 eV. Використання ван-дер-Ваальсової поправки rVV10 наближає розрахункові результати до  експериментальних значень енергетичної щілини  $\approx$ 0.22 eV.