\chapter*{Висновки}
Ми разрохували електроні властивості 1T-TiX$_2$ (X = S, Se, Te) і виявили що вони мають напівметалеву природу, окрім TiS$_2$. TiTe$_2$, TiSe$_2$ мають перекриття між смугами p і d халькогену та титану відповідно. DOS збільшується біля E$_F$ при переході від S до Te, Se, це може відбуватися із-за того, що перекриття Ti $d$ та Te, Se $p$ орбіталей у даних сполуках більше ніж з S, або як ми дослідили, воно зовсьм відсутнє і це добре узгоджується за нещодавними експериментами.

Для того щоб правильно описати щілину у TiS$_2$ знадобилось додаково взяти до уваги що це сильнокорельована система та до SCAN функціоналу додати взаємодію Хаббарда. Були установлені оптимальне значення U, яке складає 2.1 eV при якому щілина відкривається приблизно $\approx$ 0.1 eV. Потім було використано ван-дер-Ваальсову поправку rVV10 і ми змогли наблизитись до експериментальних значень щілини $\approx$ 0.22 eV. 