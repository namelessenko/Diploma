\chapter{Вступ}
Після відкриття А. Геймом та К. Новосьоловим графену у 2004 році \cite{Graphene}, знову зріс інтерес до двохвимірних матеріалів, зокрема до дихалькогенідів перехідних металів (Transition metal dichalcogenides) та їх інтеркаляційних сполук, які дуже сильно досліджувались у 80-х роках. Ці сполуки мають формулу MX$_{2}$ (X = S, Se, Te; M = Ti, Zr, Hf, V, Nb, Ta, Mo, W). TMDS складаються з складених тришарових X-M-X і мають гексагональну або тригональну симетрію. Ці тришарові шари утримуються разом слабкими силами ван-дер-Ваальса, що дозволяє відшаровувати окремі тришарові шари і осаджувати ці шари на різні підкладки. Цей простий метод отримання наноструктур у поєднанні з багатими фізичними властивостями TMD роблять їх перспективними матеріалами для спінтроніки, наноелектроніки, виробництва відновлюваної енергії, біохімічних застосувань, а також для valleytronics – абсолютно нового підходу до квантових обчислень.

Ключовою особливістю, загальною для багатьох матеріалів TMD, є впорядкування хвилі щільності заряду (CDW), яке часто виникає поблизу надпровідності. Типовим прикладом є 1T-TiSe$_2$, в якому перехід у стан CDW відбувається при CDW ~ 200K. Нижче TCDW TiSe$_2$ кристалізується в форму надрешітки 2x2x2 CDW з симетрією P-3c1 (165 просторова симетрія), в той час як неспотворений кристал при T > T$_{CDW}$ має симетрію P-3m1 (164 просторова симетрія). Нижче переходу питомий опір сильно зростає.

Споріднені сполуки 1T-TiS$_2$ і 1T-TiTe$_2$ отримують з TiSe$_2$ шляхом ізовалентної заміни Te і S На Se відповідно. Сполуки мають ту ж кристалічну структуру, що і 1T-TiSe$_2$ в умовах навколишнього середовища. Однак, на відміну від TiSe$_2$, 1T-TiTe$_2$ не проявляє ніякого CDW в об'ємі при низьких температурах. Нещодавно було показано, що (2x2) CDW може з'являтися тільки в моношарах TiTe2 (P. Chen та ін.). Експериментальна ситуація з TiS$_2$ більш суперечлива. Досі обговорюється, чи має він напівпровідникову або напівметалічну природу. Частина фазової діаграми TiSe$_2$--$_x$S$_x$ до X < 0,34 була нещодавно вивчена за допомогою ARPES і STM. Автори бачили, що CDW поступово пригнічується. Екстраполюючи свої результати, автори дійшли висновку, що CDW звертається в нуль при легуванні, близькому до x = 1.

Оскільки модуляція щільності заряду та спотворення структури в стані CDW відбуваються одночасно, рушійна сила переходу CDW все ще обговорюється. Було запропоновано кілька механізмів, включаючи фононну конденсацію, екситонний механізм, зонну нестійкість Яна-Теллера, орбітальне впорядкування, взаємодія нестійкостей Купера і частинок-дірок.

Метою даної роботи є вивчення тенденції до CDW в рядку Ti(S, Se, Te)$_2$ шляхом аналізу основного стану і низькоенергетичних метастабільних станів, відповідних низькоенергетичним локальним мінімумам функціоналу Кона-Шаму. Для отримання локальних мінімумів у вищезазначених системах використовується метод еволюційного пошуку з фіксованим складом. Цей метод раніше успішно використовувався для прогнозування нових фаз шаруватих матеріалів, таких як VSe$_2$ [], PS$_2$ [], FeS$_2$ [], TiTe$_2$ під тиском [], а також для обробки 2D-систем, таких як борофен [], фаграфен [], 3D монокарбіди перехідних металів []. Для отриманих структур ми досліджуємо вплив поправки ван-дер-Ваальса до функціонала Кона-Шама на Параметри прогнозованих структур.

\section{Літературній огляд досліджуваних систем}
Дослідження електронної будови TiS$_{2}$ дали суперечливі висновки. З одного боку деякі розрахунки зонної структури приводять до непрямого перекриття p/d смуг між точками $\Gamma$ та $L$ в діапазоні від 0.2 до 1,5 еВ. Інші стверджують, що TiS$_{2}$ є вузько-щілинним напівпровідником. Експериментальні данні отримані за допомогою фотоемісії вказують на те, що поведінка даної сполуки схожа на напівпровідникову \cite{semiconducter} або ж напівметалічну \cite{semimetal}. Як відомо, TiS$_{2}$ має високу електронну провідність без зовнішнього легування, однак походження цієї високої провідності, будь то напівметал або сильно легований напівпровідник, обговорюється протягом декількох десятиліть, але деякі недавні GW, DFT розрахунки в поєднані з сканувальною зондовою мікроскопією, все ж таки, стверджують, що висока провідність обумовлена сильним самолікуванням \cite{semimetal_or_semiconducter}. 

Щодо прикладного застосування, то даний матеріал успішно використовується у літій-іонних акумуляторах у якості катода. Коефіцієнт дифузії літію у TiS$_2$ порядку $10^{-8}-10^{-7}$ см$^2$/с, на один --- два порядки вище, ніж у широко використовуваних оксидних катодів. Недавні досліди показують, що комірки TiS$_2$ могуть зберігати більш ніж 50 \% початкової ємності після 35 років зберігання. Однією з важливих причин, чому TiS$_2$ був обраний як катодний матеріал для літій-іонних батарей, є те, що він має високу внутрішню електропровідність без зовнішнього легування. Це відрізняється від деяких інших популярних катодів таких, як LiFePO$_4$, для яких низька провідність була головною проблемою.

У 2003 році метод низькотемпературного газофазного синтезу(TiCl$_4$ + 2H$_2$S $\rightarrow$ TiS$_2$ + 4HCl), був успішно використаний для отримання нанотрубок на основі TiS$_2$ \cite{nanotube}. Аналіз їх морфології і структури показав, що трубки складаються з співвісних шарів сульфіду титану (відстань між шарами становить 0,57 нм) з атомним співвідношенням Ti:S =1:2, мають відкриті кінці, середні значення зовнішнього і внутрішнього діаметрів трубок складають  20 - 30 і  10 нм відповідно. У роботах \cite{nanotube2,nanotube3} були вивчені процеси інтеркалювання нанотрубок TiS$_2$ літієм і воднем і обговорені можливості їх використання в якості матеріалів для водневих акумуляторів. Матеріалознавчі перспективи різних класів наноструктур багато в чому визначаються їх електронними властивостями, які можуть істотно відрізнятися від відповідних кристалічних (3D) фаз. Зазначені властивості в свою чергу залежать від атомної будови і геометрії наноструктур. Так, нанотрубки дисульфідів Mo, W є напівпровідниками, причому в залежності від діаметра і атомної конфігурації стінок (так званої хіральності) величина забороненої щілини різко змінюється. Навпаки, всі NbS$_2$-нанотрубки за своїми провідними властивостями є металами \cite{nanotube4,nanotube5,nanotube6}.