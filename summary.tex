\begin{center}
{\large \textbf{АНОТАЦІЯ}}
\end{center}

\textbf{Носенко А. О.} Моделювання електронної структури Ван-дер-Ваальсових матеріалів в межах DFT. \\
Кваліфікаційна робота магістра за спеціальністю 105 Прикдна фізика і наноматеріали, освітня програма Пркикладна фізика і наноматеріали. 
"--- Державна наукова установа «Київський академічний університет», кафедра прикладної фізики та наноматеріалів. "--- Київ "--- 2022.

\textbf{Науковий керівник}: д.ф.-м.н. Карбівський В. Л.

\textbf{Ключові слова}: діхалькогеніди перехідних металів, зонні розрахунки, теорія функціонала густини

\textbf{Елементи оформлення}: 13 рисунків, 4 таблиць, 1 список.

Використовуючи в рамках теорії функціонала густини підходи GGA в PBE параметризації та meta-GGA в SCAN параметризації, було визначено оптимальні параметри розрахунків електронних властивостей, зокрема визначили оптимальний параметр Хаббардовської взаємодії U, дихалькогенідів перехідних металів 1T-TiX$_2$ (X = S, Se, Te). Встановили електронну природу 1T-TiS$_2$, 1T-TiSe$_2$, 1T-TiTe$_2$ та розрахувати ширину забороненої зони для 1T-TiS$_2$. 



\bigskip

\clearpage

\thispagestyle{empty}
\begin{center}
{\large \textbf{SUMMARY}}
\end{center}

\textbf{Nosenko A. O.} Modeling of the electronic structure of van der Waals materials by means DFT. \\
Masters qualification work in specialty 105 Applied physics and nanomaterials, educational program Applied physics and nanomaterials "--- State Research Institution «Kyiv Academic University», Department of Applied physics and nanomaterials. "--- Kyiv "--- 2022.

\textbf{Research supervisor}: Doctor of Physical and Mathematical Sciences, \\ Karbivsky V. L.

\textbf{Key words}: transition metal dichalcogenides, band calculations, density functional theory

\textbf{Design elements}: 13 Figures, 4 Tables, 1 list.

Using the GGA approaches in PBE parameterization and meta-GGA in SCAN parameterization within the framework of density functional theory, optimal parameters for calculating electronic properties were determined, in particular, the optimal parameter of the Hubbard interaction U for transition metal dichalcogenides 1T-TiX$_2$. (X = s, Se, Te) was determined. We have established the electronic nature of 1T-TiS$_2$, 1T-TiSe$_2$, 1T-TiTe$_2$ and calculated the width of the gap for 1T-TiS$_2$.