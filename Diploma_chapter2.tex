\chapter{Density Functional Theory}
Шляхом публікації двох статей Хохенбергом та Коном у 1964 році~\cite{Hohenberg&Khon}, а також Коном і Шамом в 1965 році~\cite{Khon&Sham}, теорія електронної будови вийшла на зовсім новий рівень. Мабуть, найважливішим є те, що ТФГ це теорія взаємодіючої та корельованої системи електронів. Версія ТФГ Кона-Шама використовує метод незалежних частинок; вона будує допоміжну систему, яка визначає точно електронну щільність та енергію фактично взаємодіючої системи.

В ТФГ Кона-Шама всі обмінні та кореляційні ефекти включені в обмінно-кореляційний функціонал $E_{xc}[n]$, котрий залежить від густини $n(\textbf{r})$ електронів. 

\section{Підхід Кона-Шама}
Функціонал повної енергії може бути записаний, як~\cite{K-S energy}:
\begin{eqnarray}
 E[\psi_i] = 2\sum\limits_{i}\int\psi_i\left({{\hbar^2}\over{2m}}\right)\nabla^2\psi_id^3\textbf{r}+\int V_{ion}(\textbf{r})n(\textbf{r})d^3\textbf{r}+\nonumber \\
 {{e^2}\over{2}}\int {{n(\textbf{r})n(\textbf{r})}\over{|\textbf{r}-\textbf{r}^\prime|}}d^3\textbf{r}d^3\textbf{r}^\prime+E_{xc}[n(\textbf{r}^\prime)]+E_{ion}(R_i)
\end{eqnarray}

Де $V_{ion}$ -- повний електрон-іонний потенціал, $E_{xc}[n(\textbf{r}^\prime)]$ -- обмінно-кореляційний функціонал, $E_{ion}$ -- кулонівська енергія, $n(\textbf{r})$ -- електронна густина

\subsection{Теореми Кона-Шама}

\textbf{Перша теорема} говорить про те, що електронна щільність єдиним чином визначає оператор Гамільтона і, отже, всі характеристики системи.
Не можуть існувати два різні зовнішні потенціали, що призводять до однієї і тієї ж електронної щільності основного стану системи, іншими словами, електронна щільність основного стану однозначно визначається зовнішнім потенціалом.
Знаючи потенціал, ми знаємо Гамільтоніан системи, відповідно можемо знайти хвильову функцію системи та всі властивості системи, що визначаються електронною щільністю основного стану. Перша теорема показує зв'язок між зовнішнім потенціалом та електронною щільністю основного стану.

\textbf{Друга теорема} функціонал енергії, що визначає енергію квантового стану системи, визначає мінімальну енергію тоді і тільки тоді, коли електронна густина, що входить у функціонал, є реальною густиною основного квантового стану.
Таким чином, для знаходження точної енергії основного стану та його щільності, достатньо знати функціонал $E[n]$. Висновки цих теорем призводять до того, що для будь-якого зовнішнього потенціалу завжди можна знайти електронну густину та енергію основного стану, мінімізуючи цей функціонал.


\subsection{Самоузгоджене рівняння Кона-Шама}
Електронна щільність системи може бути знайдена з рішення самоузгодженого рівняння Кона-Шама, це випливає з наведених раніше теорем, яке можна записати так:

\begin{equation}
    \left[{{-\hbar^2}\over{2m}}\nabla^2+V_{ion}(\textbf{r})+V_H(\textbf{r})+V_{XC}(\textbf{r})\right]\psi_i(\textbf{r}) = e_i\psi_i(\textbf{r})
\end{equation}

Де $\psi_i$ -- хвильова функція стану $i$, $e_i$ -- власні значення, $V_H$ -- потенціал Хартирі, який відповідає за електронно-електронне відштовхування.

Явний вид обмінно-кореляційного потенціалу $V_{XC}$ ми не можемо знати принципово, тому його задають формально, функціональною похідною: 

\begin{equation}
    V_{XC} = \frac{\partial E_{XC}[n(\textbf{r})]}{\partial n(\textbf{r})}
\end{equation}

Для визначення об'ємно-кореляційного потенціалу використовують ряд наближень, про які ми поговоримо у наступному розділі.


\section{Методи апроксимації \textbf{$E_{xc}$}}
\subsection{LDA}
LDA (local density approximation) -- це клас наближень для обмінно-кореляційної енергії $E_{XC}$ у ТФГ, яка залежить виключно від електронної густини в кожній точці простору. Найуспішнішими локальними наближеннями є ті, що були отримані з моделі однорідного електронного газу (HEG)~\cite{Hohenberg&Khon}. 

Можна записати наступний вираз для обмінно-кореляційної енергії:

\begin{equation}
    E_{XC}^{LDA}[n(\textbf{r})] = \int{\epsilon_{xc}[n(\textbf{r})]n(\textbf{r})d^3\textbf{r}}
\end{equation}

Існує ряд параметрів для LDA, але ми їх не розглядатимемо.

\subsection{GGA}
GGA (generalized gradient approximations) -- на відміну від LDA в даний клас наближень включено градієнтну поправку для електронної щільності. Це усуває деякі недоліки LDA. Для узагальненого градієнтного наближення ми можемо записати, що $E_{xc}$ дорівнює деякій функції локальної щільності та її градієнта:

\begin{equation}
    E_{xc}^{GGA} = \int{f[n(\textbf{r}),\nabla{n(\textbf{r})]n(\textbf{r})d^3\textbf{r}}}
\end{equation}

Найчастіше для розрахунків використовуються параметризації PBE (Perdew-Burke-Ernzerhof)~\cite{PBE} та PW91~\cite{PW91}.

\subsection{meta-GGA}
Фактично meta-GGA - це розширене наближення GGA. В яке, як вхідні дані, входить щільність позитивної орбітальної кінетичної енергії~\cite{Swapan&Gosh&Parr, Becke&Roussel, Tao&Perdew}.

У напівлокальному наближенні функціонал $E_{xc}$ зводиться до єдиного інтегралу загального вигляду:
\begin{equation}
    E_{XC}^{MGGA} = \int{\epsilon_{xc}[n(\textbf{r}),\nabla{n(\textbf{r}),\tau_{\sigma}(\textbf{r}})]}n(\textbf{r})d^3\textbf{r}
\end{equation}

Де $\tau_{\sigma}(\textbf{r})$ щільність кінетичної енергії зайнятих станів та визначається як:
\begin{equation}
    \tau_{\sigma}(\textbf{r}) = \sum\limits_{i\textbf{k}}|\nabla\psi_{i\textbf{k}}(\textbf{r})|^2
\end{equation}
\subsubsection{SCAN}
SCAN (Strongly-constrained and appropriately-normed) meta-GGA, раніше розроблені meta-GGA функціонали виявилися менш точними для розрахунку критичних тисків у структурно-фазових переходах твердих тіл, а також для матеріалів з шаруватою структурою відомі, як Ван-дер-Ваальсовські матеріали. SCAN покликаний усунути цю проблему за допомогою додавання в обмінно-кореляційний функціонал безрозмірну змінну.
$\alpha$~\cite{SCAN}:
\begin{equation}
    \alpha = (\tau - \tau^W)/\tau^{unif} > 0
\end{equation}

Де $\tau^W = |\nabla{n(\textbf{r})}|^2/8n(\textbf{r})$ є одноорбітальною межею $\tau$ та $\tau^{unif} = (3/10)(3\pi^2)^{2/3}n(\textbf{r})^{5/3}$ межа рівномірної щільності.
\subsubsection{SCAN+rVV10}
Складність опису "шаруватих" матеріалів виникає через нелокальну далекодійну природу ван дер Ваальсовської взаємодії, а також через її більш слабкий зв'язок в порівнянні з хімічною. Повний облік взаємодії Ван дер Ваальса досягається такими дорогими методами, як Монте Карло (QMC)~\cite{QMC}, одиничні та подвійні зв'язані кластери з перетурбованими трійками (CCSD(T))~\cite{CCSD(T)} та інші, через їхню дорожнечу їх використовують тільки на строго обмежених системах. Поправка rVV10 має такі параметри як $b$, який відповідає за короткодійну поведінку нелокальної кореляційної енергії, а також $C$, що відповідає за так звану локальну заборонену зону. Ці параметри фітуються на S22. Набір даних тесту S22~\cite{S22} використовується для оцінки відносної точності методів vdW-DF, також обговорюються такі фактори, як масштабованість та переносність.